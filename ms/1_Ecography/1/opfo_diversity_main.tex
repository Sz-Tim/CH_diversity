%% This is file `elsarticle-template-1-num.tex',
%%
%% Copyright 2009 Elsevier Ltd
%%
%% This file is part of the 'Elsarticle Bundle'.
%% ---------------------------------------------
%%
%% It may be distributed under the conditions of the LaTeX Project Public
%% License, either version 1.2 of this license or (at your option) any
%% later version.  The latest version of this license is in
%%    http://www.latex-project.org/lppl.txt
%% and version 1.2 or later is part of all distributions of LaTeX
%% version 1999/12/01 or later.
%%
%% Template article for Elsevier's document class `elsarticle'
%% with numbered style bibliographic references
%%
%% $Id: elsarticle-template-1-num.tex 149 2009-10-08 05:01:15Z rishi $
%% $URL: http://lenova.river-valley.com/svn/elsbst/trunk/elsarticle-template-1-num.tex $
%%
\documentclass[preprint,review,times,12pt,3p]{elsarticle}

%% Use the option review to obtain double line spacing
%% \documentclass[preprint,review,12pt]{elsarticle}

%% Use the options 1p,twocolumn; 3p; 3p,twocolumn; 5p; or 5p,twocolumn
%% for a journal layout:
%% \documentclass[final,1p,times]{elsarticle}
%% \documentclass[final,1p,times,twocolumn]{elsarticle}
%% \documentclass[final,3p,times]{elsarticle}
%% \documentclass[final,3p,times,twocolumn]{elsarticle}
%% \documentclass[final,5p,times]{elsarticle}
%% \documentclass[final,5p,times,twocolumn]{elsarticle}


%%\usepackage[left=1in, right=1in, top=1in, bottom=1in]{geometry}
\usepackage{array}
\usepackage{multirow}
\usepackage{graphicx}
\usepackage{amssymb}
\usepackage{amsthm}
\usepackage{amsmath}
\usepackage{lineno}
\usepackage{setspace}
\usepackage{pdflscape}
\usepackage{natbib}

\journal{Ecological Modelling}
\setcitestyle{authoryear,round,semicolon,sort}
\bibliographystyle{apalike}
\begin{document}
\begin{frontmatter}

\title{Leveraging citizen science to assess richness, diversity, and abundance}

\author[DEE]{Tim M. Szewczyk}
\author[DEE]{Cleo Bertelsmeier}
\author[DEE]{Tanja Schwander}

\address[DEE]{Department of Ecology and Evolution, University of Lausanne}


\begin{abstract}
Abstract.
\end{abstract}

\begin{keyword}
Keywords
\end{keyword}

\end{frontmatter}
\linenumbers



\section{Introduction}
\label{S:1}

% Growth of citizen science
Citizen science holds strong potential for overcoming many of the logistical complications that hinder ecological research at broad scales. The many volunteer collectors can sample ecological systems across regional, continental, and even global extents \citep{Theobald2015}, and include private land that is otherwise inaccessible to researchers \citep{Pernat2020}. Accordingly, citizen science initiatives have become increasingly common, with applications ranging from monitoring the spread of an invasive rabbit in Australia \citep{Roy-Dufresne2019} to identifying the impacts of hunting on game species' life history traits in the Amazon \citep{ElBizri2020}, and the samples collected by citizen scientists represent a sizable component of many ecological data sets \citep{Poisson2020}.  While citizen science effort has not been equal across taxonomic groups \citep{Theobald2015,Troudet2017}, such projects could be particularly useful for overcoming the shortfall in knowledge of insect distributions, since specimens can be collected and easily returned for expert identification or genetic analysis \citep{Pernat2020}.

% Value and potential shortcomings
The effort of citizen scientists represents a valuable resource for understanding and monitoring the distribution of biodiversity on a rapidly changing planet. Unfortunately, despite the explosion in citizen science initiatives, as few as 12\% are associated with peer-reviewed publications \citep{Theobald2015}. Datasets collected by citizen scientists do have limitations, even with relatively straightforward designs such as collecting unstructured species occurrences for use in species distribution models \citep{Theobald2015,Steen2019,Duan2020,Henckel2020}. The typically decentralized nature of the data collection may require additional quality checks and filters, consideration of the potential for errors in species identifications or geo-location, and handling of non-random sampling. However, advances in modelling provide viable solutions to compensate for these pitfalls. For example, after accounting for such issues as spatial bias \citep{Isaac2014,Steen2019,Johnston2020,Robinson2020}, the performance of species distribution models fit with citizen science data can be equivalent or superior to those fit with scientifically collected data \citep{Steen2019,Sumner2019,Henckel2020}. 

% Modelling richness with presence-only data: iSDMs, jSDMs, MRMs, mSDMs
Survey-type citizen science projects often collect a broad set of species within a taxon, making them appealing to use for estimating taxon-level patterns of richness. A variety of methods have been developed, broadly categorized into those that model richness directly as a function of environmental variables, and those grew from methods for modelling individual species distributions \citep{Dubuis2011,Guisan2011,Calabrese2014,Biber2019}. For the latter, a simple approach stacks independent species distribution models, while more complex approaches model species' distributions jointly or in a hierarchical framework \citep{Caradima2019}. Alternatively, multi-species occupancy models explicitly account for imperfect detection and predict the underlying occupancy state and probability of occurrence of each species in the community \citep{Frishkoff2019,Guillera-Arroita2019,Szewczyk2018,Devarajan2020}. However, occupancy models require presence-absence data collected across multiple sampling bouts to estimate detection probabilities, and consequently their use is largely limited to sampling designs with repeated samples.

% Methods for merging data sets
Recent data integration techniques allow researchers to capitalize on the strengths of unstructured presence-only data sets while overcoming their limitations by combining them with structured samples \citep{Isaac2019,Miller2019}. These methods aim to exploit the respective strengths of unstructured presence-only data sets such as those collected by citizen scientists, and structured samples collected by researchers. For example, presence-only citizen science data are often better at detecting rare species, while data sets collected in a structured manner better represent community composition and abundance \citep{Steen2019,Henckel2020,Pernat2020}. Models based on these integrated data sets often outperform those based on the individual data sets \citep{Dorazio2014b,Fithian2015,Koshkina2017a}. Bayesian methods naturally and flexibly accommodate the incorporation of multiple data sets \citep{Clark2005,BeckEtAl2012,Szewczyk2018}, either by estimating a joint likelihood for all observations, or by incorporating the unstructured presence-only data set as a covariate or via prior distributions \citep{Fletcher2019,Isaac2019,Miller2019}. 

% Poisson point process models to unite spatial scales and accommodate density/abundance data
A natural choice for modelling species distributions or suitability across landscapes are inhomogenous Poisson Point Process Models (PPMs), which model a continuous, spatially variable intensity that produces Poisson-distributed occurrences \citep{Renner2013a,Renner2015}. PPMs are an especially appealing option for combining different data sets. Observations collected with different methods can be modelled using appropriate sampling submodels based on the same underlying intensities \citep{Fithian2015,Hefley2016,Koshkina2017a,Fletcher2019,Renner2019}, and the intensities can be integrated to the appropriate resolution, allowing patterns and processes to be linked across spatial scales \citep{Keil2014,Hefley2016}.

% Mission statement
Here, we combine occurrences of ants collected in a citizen science project in the canton of Vaud, Switzerland with species abundance data from a concurrent structured sampling effort. In a hierarchical Bayesian framework, this community-level PPM models local ant colony density, with additional presence-only information integrated at a regional scale, while accounting for taxonomy and correlated residuals among species resulting from unmeasured variables or biotic interactions. We use this model to predict the patterns of ant colony density, richness, and diversity across the landscape at local and regional scales while incorporating uncertainty in species compositions, and to evaluate the support for hypothesized environmental drivers across spatial scales. We compare inferences from the combined model with those from only the structured data set, and also assess the differences in observed communities from each sampling method. 





\section{Methods}
\label{S:2}
\subsection{Study region \& sampling design}
This study focuses on ant species in the canton of Vaud in western Switzerland (46.2–47.0$^{\circ}$N, 6.1–7.2$^{\circ}$E), a topographically heterogeneous region (372–3201m) that includes the Jura mountains in the west, the Alps in the east, and the Swiss plateau centrally (Fig. \ref{fig:VD_map}). Much of the central plateau is dominated by anthropogenic land use, including cropland, vineyards, pastures, and urban zones, with forested areas that are predominantly deciduous. The mountain ranges rise steeply from the plateau from approximately 1000m and span the forested montane and subalpine zones through the alpine zone, with widespread cattle grazing \citep{Delarze2015,Gago-Silva2017,Beck2017}. 

\begin{figure}
	\centering\includegraphics[width=3.5in]{../../../ms/1_Ecography/1/figs/map_VD+inset.png}
	\caption{\label{fig:VD_map} Map of the canton of Vaud, Switzerland. Samples consisted of presence-only data (blue points) aggregated to a 1 km$^2$ grid (light gray) and structured abundance data collected at 1 km$^2$ long-term biodiversity monitoring sites (red squares). }
\end{figure}

% Methods for citizen science data
Ants were collected during the summer of 2019 in two simultaneous collection efforts (Fig. \ref{fig:VD_map}), comprising a presence-only data set and a structured abundance data set. For the presence-only data set, a citizen science project (Opération Fourmis: https://wp.unil.ch/fourmisvaud/) was organized to survey the ant fauna within the canton of Vaud. Vials of ethanol were distributed to interested citizen scientists, who were asked to collect approximately 10 ant workers per colony for each vial. Participants were encouraged to explore under rocks, on bark, inside twigs, and in downed wood. An online map was updated periodically to highlight data-sparse areas. Collectors returned the vials along with the collection date and the locality of the sample including the latitude and longitude. Additionally, several guided collecting field trips were organized throughout the season. For these presence-only data, we discretized the landscape into a 1 km$^2$ grid (3,558 km$^2$) and tallied the number of occurrences for each species in each cell. 

% Methods for structured samples
For the structured abundance data set, local ant densities were estimated within 44 sites of 1 km$^2$ each, with standardized effort across sites. Thirty-nine of the sites were established governmental biodiversity monitoring sites, and as such were arranged on a regular grid with approximately 5-7 km between adjacent sites. Five additional sites are established monitoring sites by the University of Lausanne. The ant densities at each site were characterized by 25 plots, distributed among 15 habitat types \citep{Gago-Silva2017} in approximate proportion to the abundance of each habitat, where each habitat type present within the site was represented by at least one plot. Inaccessible areas (e.g., cliffs, water, property beyond Vaud) were excluded, resulting in several sites with areas less than 1 km$^2$, and the number of plots was reduced proportionally (1059 total plots). 

Each sampling plot consisted of a 2 m radius circle, with soil temperature recorded in the center approximately 6 cm deep, and vegetation characterized according to Braun-Blanquet coverage classes for grass, forb, shrub, litter, bare, and moss within the plot \citep{Douglas1978}. Six flags were evenly spaced around the circumference. Within $\sim$ 25 cm of each flag (total surface area $\sim$ 1.2 m$^2$), we searched for ant colonies within any downed wood or stumps, under large rocks, and in 2 L of soil, litter, and small rocks using 18 cm Hori Hori gardening knives. We haphazardly collected 10 workers from each colony, placing them directly into vials of ethanol. Within each plot, all trees $\geq$3 cm diameter at breast height were also inspected for ant workers which were collected regardless of whether or not a colony was identifiable. Lastly, transect lines were mapped \emph{a priori}, distributed proportionally among habitat types and totalling 2 km. Transects were surveyed at a moderate pace, with workers collected from all permanent above-ground mounds within 2 m of the transect line. Because of the differing sampling methods, the tree and transect collections were incorporated into the presence-only data set. 

Thus, the two data sets characterize the regional ant fauna in distinct ways, each with strengths and limitations. The presence-only data set was collected very broadly spatially, and included collections from free investigation of subjectively suitable habitats, and was therefore more likely to include rare or secretive species. However, many collections from citizen scientists occurred in or near human-dominant areas, with an expected bias toward larger, more obvious, or more anthropophilic species \citep{Ward2014, Troudet2017}. The sampling effort was also quite heterogeneous (Fig. \ref{fig:VD_map}). In contrast, the structured abundance data set was collected with uniform sampling effort, with a sampling design aimed to produce representative samples of colony density within each site. The communities from this data set are thus expected to be more representative of the ground-nesting ant community structure. 

% Identification and storage
All ants were identified to species or species group based on morphology by taxonomists. Four genera (\emph{Camponotus}, \emph{Tapinoma}, \emph{Temnothorax}, \emph{Tetramorium}) were identified based on mitochondrial DNA due to uncertainties in morphological identifications in these groups. Specimens are stored at the Natural History Museum of Lausanne in Lausanne, Switzerland. 


\subsection{Environmental variables}
Environmental variables were selected based on theoretical or empirical support in the literature as likely drivers of ant richness, diversity, or species distributions \citep{Bishop2017,Liu2018,Szewczyk2018,Longino2019,Uhey2020}. At a regional (1 km$^2$) scale, we included growing degree days from ENVIREM as well as its square \citep{Title2018}, annual precipitation from CHELSA \citep{Karger2017}, net primary productivity as the average values across years 2010–2019 (MODIS: MOD17A3), the Shannon diversity of land cover types based on proportional composition \citep{Gago-Silva2017}, the coverage of edge, forest, and agricultural habitat types \citep{Gago-Silva2017}, the average north-facing aspect calculated as the cosine of the aspect in degrees based on the ASTER digital elevation model \citep{Tachikawa2011}, the log-transformed total length of roads \citep{OpenStreetMap}, and the log-transformed total perimeter of buildings \citep{OpenStreetMap}. All variables were summarised as the mean or total value within each 1 km$^2$ grid cell and within each 1 km$^2$ structured site.

Local variables were collected in the field at each structured sampling plot as described above. We included relative soil temperature as a measurement of local variation in temperature, where the recorded temperatures among plots were z-transformed within each site. Canopy was classified as open, closed, or mixed according to land cover type (Appendix 1 Table S2). Open habitats were categorized as pasture, crop, or other according to field observations. Finally, local productivity was quantified by using the midpoint of the range of percent cover values in each Braun-Blanquet category and summing across grass, forb, and shrub in each plot \citep{Douglas1978,Mccain2018,Szewczyk2018}.

Variable processing and summarising was performed in R 3.6.3 \citep{R-3-6-3}, with spatial computations performed using the R packages \emph{raster} 3.3.7 and \emph{sf} 0.9.4. All variables were z-transformed after summarising to the appropriate spatial scale to improve model behavior.


\subsection{Model overview}
To leverage the strengths of each data set, we used a community-level hierarchical inhomogenous Poisson point process model (PPM) fit in a Bayesian framework. Inhomogenous Poisson PPMs assume that the distribution of occurrences is dependent on the variation in local intensity, $\lambda(x)$, across space, $x$, which may be observed imperfectly resulting in a thinned point process \citep{Warton2010,Baddeley2015,Fithian2015}. One key benefit of PPMs is that the underlying latent intensity is continuous in space, and can be integrated to arbitrary spatial resolutions \citep{Renner2015,Hefley2016,Koshkina2017a,Fletcher2019}. Following the structure of the sampling design, we modelled the expected intensity of each species at two resolutions (regional: 1 km$^2$, local: 1.2 m$^2$), representing the area of the sampling sites and the area of the sampling plots respectively. We modelled local intensities as a function of species' responses to the local and regional environment, with the structured abundance data incorporated at the local scale, and the presence-only data incorporated at the regional scale. The sampling submodel for each data type reflects the collection methods \citep{Isaac2014,Hefley2016,Fletcher2019,Miller2019}.


\subsection{Model structure}
The hierarchical PPM integrates the two data sets, \textbf{W} (presence-only citizen science collections; resolution: 1 km$^2$) and \textbf{Y} (structured local abundance collections; 1.2 m$^2$ plots within 1 km$^2$ sites) to predict ant communities across the landscape. Species' responses to the regional environment are informed by both \textbf{W} and \textbf{Y}, while responses to the local environment are informed by \textbf{Y}. Similarly, \textbf{W} helps to identify overall relative abundance among species, while \textbf{Y} is a direct measurement of local abundance. We assume that local colony density for each species is a function of local and regional environmental conditions, with potential phylogenetic conservatism among species-specific responses and potential residual correlation among species.

The structured abundances, \textbf{Y}, are observed at the local scale. For each species $s=1 \dots S$  at plot $i$, the number of observed colonies $Y_{is}$ follows a generalized Poisson distribution with the latent colony intensity $\lambda_{is}$ and dispersion term $\theta$ to account for overdispersion \citep{Consul1992,Ntzoufras2005,Isaac2019,Miller2019}. The local intensity $\lambda_{is}$ is a function of the local environment and the regional intensity of species $s$ at the encompassing 1 km$^2$ site $j$:
    \begin{equation}
        Y_{is} \sim GPoisson(\lambda_{is}, \theta) \\
        \label{eq:Y_GP}
    \end{equation}
    \begin{equation}
        log(\lambda_{is}) = a_s V_i + \gamma_s \zeta_i + log(h\Lambda_{js}) \\
        \label{eq:lambda}
    \end{equation}
where \textbf{V} is a matrix of local environmental covariates, $a_s$ is a vector of species-specific responses, $\zeta_i$ is a latent variable representing unmeasured local environmental variables or biotic interactions with $\gamma_s$ as a species-specific latent variable coefficient constrained between $-1$ and $1$ \citep{Ovaskainen2016,Caradima2019,Tobler2019}, $h$ is a constant scaling factor representing the proportion of site $j$ sampled by plot $i$ (1.2 $m^2 /$ 1 km$^2$ = 1.2e-6), and $\Lambda_{js}$ is the regional intensity of species $s$ at site $j$. Thus, $log(h\Lambda_{js})$ functions as a site-level intercept, determining the baseline expected intensity at each plot within each site, with local intensities then modified by effects of the local environment \citep{Yamaura2016,Miller2019}.

The presence-only data, \textbf{W}, are observed as the number of detections of each species within 1 km$^2$ grid cells (i.e., regional scale). The number of detections of each species $s=1 \dots S$ within each 1 km$^2$ cell $k$ follow a multinomial distribution such that:
    \begin{equation}
        \{W_{ks}\}_{s=1}^{S} \sim Multinomial(\sum_{s=1}^{S} W_{ks} , \varsigma_k(\{log(\Lambda_{ks})D_s\}_{s=1}^{S})) \\
        \label{eq:W_Mn}
    \end{equation}
where $\{W_{ks}\}_{s=1}^{S}$ is a vector of length $S$ with the number of detections for each species within cell $k$, $\sum_{s=1}^{S} W_{ks}$ is the total number of detections within the cell, $D_s$ is the proportional detection bias for each species, which accounts for bias in the community composition based on species that are more likely to be observed, and $\varsigma_k(\cdot)$ is the softmax function which converts the weighted intensities to probabilities that sum to 1 within each cell. For each species $s$ in cell $k$, the observed count $W_{ks}$ is expected to be higher if the species has high relative abundance ($\Lambda_{ks}$ is larger relative to other species), more samples were collected ($\sum_{s=1}^{S} W_{ks}$ is larger), or if species $s$ is likely to be over-represented in the presence-only data set ($D_s$ is larger). The presence-only counts thus inform the relative patterns of species across the landscape, while accounting for spatial and taxonomic bias in sampling intensity \citep{Isaac2014}.

While \textbf{W} and \textbf{Y} are each modelled with distinct sampling submodels, both $\Lambda_{js}$ and $\Lambda_{ks}$ represent the latent intensity of species $s$ at an identical 1 km$^2$ regional scale. Consequently, the regional intensity $\Lambda_{(j,k)s}$ is a shared parameter that links the two data sets \citep{Hefley2016,Isaac2019,Miller2019}. The ecological processes driving variation in regional intensity are assumed to be the same across data sets, and the intensities are modelled together in a single regression such that:
    \begin{equation}
        log(\Lambda_{(j,k)s}) = b_s X_{(j,k)} \\
        \label{eq:LAMBDA}
    \end{equation}
where $b_s$ is a vector of species-specific slopes, and \textbf{X} is a matrix of regional environmental covariates. For each species, the average regional intensity (i.e., intercept) as well as variation across regional environments are informed by $Y_{is}$ linked through Eq. \ref{eq:Y_GP}-\ref{eq:lambda}, with the relative variation across regional environments further informed by $W_{ks}$ linked through Eq. \ref{eq:W_Mn}.  

The slopes $a_s$ and $b_s$ are species-specific responses at local and regional scales, respectively, and are distributed about genus-level means $A_g$ and $B_g$ with standard deviations $\sigma_a$ and $\sigma_b$. The genus-level means are in turn distributed about aggregate means, $\alpha$ and $\beta$, with correlation matrices $\Sigma_A$ and $\Sigma_B$, which reflect the overall responses of the ant community to environmental variables at each resolution while accounting for relatedness at the genus level \citep{Hadfield2010b,Ovaskainen2011,Szewczyk2018,Caradima2019}.

Prior distributions were lightly informative to constrain the sampling algorithm to plausible ranges \citep{Carpenter2017,Lemoine2019}. Specifically, aggregate slopes $\alpha$ and $\beta$ were distributed Normal($\mu=0$, $\sigma=1$), the log intensity intercept $\beta_0$ was distributed Normal($\mu=-4$, $\sigma=2$), and standard deviations for species and genus slopes were distributed Normal($\mu=0$, $\sigma=2$) and constrained to be positive. The correlation matrices used a Cholesky decomposition and followed an LKJ prior as recommended for better model performance \citep{Carpenter2017,Caradima2019} with $\eta=2$. The proportional taxonomic bias term $D$ was distributed Normal($\mu=1$, $\sigma=1$) and constrained to be positive. Code for the full model is available on GitHub (https://github.com/Sz-Tim/CH\_diversity -- versioned DOI at submission).

Finally, several quantities were calculated in each sample from the posterior. Probability of presence was calculated as $1 - e^{-\Lambda_{(j,k)s}}$ at the regional scale \citep{Hefley2016}, and $1 - pr(0 | \lambda_{is}, \theta)$ at the local scale. Predicted richness was calculated within each plot or grid cell as the sum of species-level Bernoulli draws based on the probability of presence. Shannon's H was calculated at each plot using $\lambda_{i,1:S}$ and in each cell using $\Lambda_{(j,k),1:S}$. Total predicted intensity was calculated as $\sum_{s=1}^{S}\Lambda_{(j,k)s}$ and $\sum_{s=1}^{S}\lambda_{is}$. The correlation in residuals among species was calculated as $\gamma \gamma^T + diag(1 - \gamma^2)$ \citep{Tobler2019}. Posterior predictions of local patterns were based only on the local plots at the 44 structured sites, while those of regional patterns were based on the 44 structured sites and the entire gridded landscape for Vaud (Fig. \ref{fig:VD_map}). 


\subsection{Variable selection \& model fitting}
To evaluate the effect of including the presence-only data, \textbf{W}, we compared a version of the model with and without \textbf{W}. We refer to the model that uses both \textbf{W} and \textbf{Y} as the \emph{Joint} model, and the model using only \textbf{Y} as the \emph{Structured} model. The \emph{Structured} model therefore does not include Eq. \ref{eq:W_Mn}, but nevertheless includes species only detected in \textbf{W}. We performed variable selection separately on the two models.

Because comparing all possible models was not feasible, variable selection was performed using a 4-fold cross-validation with a step-wise forward search. The 44 sites in the structured abundance data set \textbf{Y} were divided into 4 subsets using the R package \emph{caret} 6.0.86. The models were parameterized using 3 of the subsets and predictive ability assessed using the 4$^{th}$, with four separate model runs such that all subsets were predicted once. In each round, the log pointwise predictive density of the observed abundances given the predicted local intensities for each candidate model was calculated and compared using the R package \emph{loo} 2.3.1. Variables were added sequentially, and the variable sets for the \emph{Joint} and \emph{Structured} models with the best out-of-sample predictive ability were considered optimal. Using the optimal variable sets, the models were fit using all observations, with intensities for each species then predicted across the full landscape, excluding cells with covariates more than three standard deviations beyond the values in the cells with data to avoid unwarranted extrapolation.

We assessed model performance of the optimal models based on deviance using the structured abundance observations such that $d = \sum_{i=1}^{I}\sum_{s=1}^S log(P(Y_{is} | \lambda_{is}, \theta))$ in each sample from the posterior, with the explanatory power of each proposed model as $D^2 = \frac{d_{null} - d_{proposed}}{d_{null}}$ \citep{Caradima2019,Guisan2000}. We used an intercept-only model fit with the structured abundances \textbf{Y} as the null model. We calculated an overall $D^2$ using all abundances, species-specific $D^2_s$ to compare performance across species, and plot-specific $D^2_i$ to compare performance across elevations. We calculated these metrics for the fitted full model, and for the predicted subsets during the 4-fold cross validation to assess improvement in performance for out-of-sample plots in the optimal models.

All models were written in Stan \citep{Carpenter2017} and run in \emph{CmdStan} 2.23.0. In contrast to other commonly used Bayesian software such as BUGS or JAGS, Stan uses a Hamiltonian Monte Carlo sampling algorithm which requires a warm-up period to tune the sampler parameters rather than a burn-in period. During variable selection, we ran 3 chains for each model with 2,500 iterations per chain, with 2,000 iterations for the warm-up period. For the optimal models, we ran 12 chains with 2,250 iterations each, with 2,000 iterations for the warm-up period. Model behavior and convergence was assessed by confirming that all $\hat{R}$ values were $\leq$ 1.1, visually inspecting a selection of trace plots including hyperparameters and scale parameters, and confirming that no divergent transitions occurred during the sampling, which would indicate a poor approximation of the joint posterior distribution. Highest Posterior Density Intervals (HPDIs) were calculated for all parameters of interest using the R package \emph{HDInterval} 0.2.2.


\subsection{Community analyses}
Using the $\lambda$ posterior medians, we calculated the $\beta$-diversity among plots within each site using the R package \emph{betapart} 1.5.1. The overall $\beta$-diversity at each site was partitioned into the balanced variation and abundance-gradient components, representing changes in relative abundance among species and changes in total abundance, respectively \citep{Baselga2017}. Change across elevation for total $\beta$-diversity and each component was assessed using linear and quadratic regressions compared with AICc using the R package \emph{AICcmodavg} 2.3.0. When $\Delta$AICc was $<$ 4, the linear model was selected as most parsimonious. 

A double principal coordinates analysis (DPCoA) was also performed using $\lambda$ posterior medians to assess similarity in taxonomically-weighted community structure among plots \citep{Dray2015,Pavoine2019} using the R packages \emph{ade4} 1.7.15 and \emph{vegan} 2.5.6. The elevational distribution of each species was characterized by the observed elevational range, and by the proportion of total $\Lambda$ in plateau ($<$1000m) or montane ($\geq$1000m) environments.




\section{Results}
\label{S:3}
A total of 79 species were detected across both data sets, with 76 in the presence-only data set (\textbf{W}) and 51 in the structured abundance data set (\textbf{Y}). The presence-only data set contained 6632 samples collected by citizen scientists and 363 supplemental transect or tree samples, covering nearly half of the 1 km$^2$ grid (1309 cells with occurrences = 42\%; range: 1–162 occurrences; median: 3 occurrences). The structured abundance data set contained 1090 colonies detected across the 44 sites (per-site median: 22.5 colonies; range: 6–58 colonies). 

% Model performance & covariates
The \emph{Joint} model, parameterized with both the \textbf{W} and \textbf{Y}, consistently outperformed the \emph{Structured} model, parameterized with only \textbf{Y}, in predicting novel local communities during cross-validation (optimal models: \emph{Joint} $D^2$=0.079; \emph{Structured} $D^2$=0.050; Appendix 2 Table S5). However, in the full model, the \emph{Structured} model fit the data somewhat more closely (\emph{Joint} $D^2$=0.19; \emph{Structured} $D^2$=0.21). This same pattern occurred on average across species (\emph{Joint} $D^2_s$ median: 0.16; \emph{Structured} $D^2_s$ median: 0.21) and across plots (\emph{Joint} $D^2_i$ median: 0.27; \emph{Structured} $D^2_i$ median: 0.28). There was no elevational pattern in $D^2_i$.

The optimal variables for both models included growing degree days at a regional scale, and local effects of relative soil temperature and vegetation cover (Fig. \ref{fig:slope_means}a). The optimal \emph{Joint} model also included regional effects of forest cover and road length, while the optimal \emph{Structured} model included local effects of canopy type. For many variables, species showed a variety of responses, resulting in no clear aggregate effect as the 95\% Highest Posterior Density Intervals (HPDIs) for $\beta$ included zero (Fig. \ref{fig:slope_means}a, horizontal points and lines; Appendix 2 Fig. \ref{fig:b_byParam}). However, the aggregate effect of growing degree days was positive in the \emph{Joint} model, and the aggregate effect of squared growing degree days was negative in both models. For relative soil temperature, vegetation cover, and mixed canopy, nearly all species with non-zero HPDIs showed positive responses (Appendix 2 Fig. \ref{fig:b_bars}). The \emph{Joint} model reduced uncertainty in species' responses compared to the \emph{Structured} model across regional covariates, as indicated by narrower 95\% HPDIs (Fig. \ref{fig:slope_means}b). This reduction occurred on average across all species, but was greater for species that did not occur in the structured abundance data set \textbf{Y}. In contrast, uncertainty was somewhat higher for local variables.

\begin{figure}
\centering\includegraphics[width=6in]{../../../ms/1_Ecography/1/figs/slope_means+HDI.png}
\caption{\label{fig:slope_means} Covariate effects in optimal models. (a) Density curves summarise species-level posterior means for responses to variables in the model using both data sets (\emph{Joint}: purple) and the model using only the abundance data (\emph{Structured}: green). Points and lines show posterior means and Highest Posterior Density Intervals (HPDI; thick: 80\%; thin: 95\%) for the aggregate ($\beta$) responses. (b) Change in 95\% HPDI widths in species responses between models (negative: less uncertainty in \emph{Joint} model) for species observed in the structured abundance data set \textbf{Y} (dark blue) or only in the presence-only data set \textbf{W} (light blue). Points and lines show mean $\pm$ standard error across species. }
\end{figure}

% Elevational patterns
Across the canton of Vaud, predicted ant richness in both models showed a peak just below 1000m at a regional scale, with no clear trend at a local scale for the plots within the 44 structured sites (Fig. \ref{fig:el_patterns}a). In contrast, the \emph{Joint} model predicted higher regional Shannon diversity at lower elevations, while the \emph{Structured} model predicted little elevational trend (Fig. \ref{fig:el_patterns}b). At a local scale, the \emph{Structured} model predicted lower Shannon diversity at low elevations, increasing until 1000m, while the \emph{Joint} model predicted little trend. The \emph{Structured} model predicted higher regional colony intensities at lower elevations, in contrast to the increase in intensity above 1000m predicted by the \emph{Joint} model (Fig. \ref{fig:el_patterns}c). Local predicted intensity was similar across models, with little trend. 

\begin{figure}
	\centering\includegraphics[width=6in]{../../../ms/1_Ecography/1/figs/el_patterns.png}
	\caption{\label{fig:el_patterns} Elevational patterns of posterior distributions at regional and local scales for (a) species richness, (b) Shannon H diversity, and (c) total colony intensity. Lines and ribbons are loess lines using the posterior medians and 95\% Highest Posterior Density Interval, respectively, for the model parameterized with both data sets (\emph{Joint}: purple) and the model parameterized with only the structured abundance data (\emph{Structured}: green). }
\end{figure}

Both the presence-only (\textbf{W}) and structured abundance (\textbf{Y}) data sets showed strong differences in the genus composition between the plateau and montane samples (Fig. \ref{fig:genus_assemblages}). In both data sets, the plateau ($<$1000m) was dominated by \emph{Lasius spp.}, with high representation of \emph{Formica spp.} in the montane zone ($\geq$1000m). The structured abundance data set also showed high relative abundance of \emph{Myrmica spp.}, particularly in montane environments. Compared to the structured abundance data set, the presence-only data set under-represented many species (95\% HPDIs $<$ 0; \emph{Joint}: 23 species = 29\%; Appendix 2 Fig. \ref{fig:D}), including nearly every species in the genus \emph{Myrmica}, a genus particularly prevalent in the montane structured abundance samples. In contrast, only the anthropophilic pavement ant \emph{Tetramorium immigrans} was clearly over-represented (95\% HPDIs $>$ 0; \emph{Joint}: 1 species = 1\%; Appendix 2 Fig. \ref{fig:D}). 

\begin{figure}
	\centering\includegraphics[width=3in]{../../../ms/1_Ecography/1/figs/genus_assemblages.png}
	\caption{\label{fig:genus_assemblages} Genus composition in the presence-only and structured abundance data sets across plateau and montane environments. Only genera that constitute $\geq 1\%$ of at least one subset are shown, with others indicated as 'Other'.}
\end{figure}

The local ant communities showed strong separation across elevations. Using the posterior median local communities predicted by the \emph{Joint} model, the DPCoA showed communities in two distinct clusters representing plateau ($<$1000m) and montane ($\geq$1000m) plots (Fig. \ref{fig:dpcoa_joint}). The DPCoA accounts for relatedness among species, with the 100m elevational bins (Fig. \ref{fig:dpcoa_joint}a) or regions (Fig. \ref{fig:dpcoa_joint}b) applied \emph{post-hoc}. No clear pattern was seen at lower elevations, with broad, partially overlapping clusters for communities from 300–700m and 700–1000m (Fig. \ref{fig:dpcoa_joint}a). Above 1000m, communities showed a gradient, with elevation increasing down the y-axis. By regions, plateau communities formed a notably broad cluster (Fig. \ref{fig:dpcoa_joint}b), indicating greater variation among local ant communities. In contrast, montane communities formed a tighter clusters, with the communities found in the western Jura mountains appearing largely as a subset of those in the taller, eastern Alps. Results were qualitatively similar for analyses using the communities predicted by the \emph{Structured} model (Appendix 2 Fig. \ref{fig:dpcoa_struc}). Correspondingly, many species were predicted to occur predominantly in plateau or in montane elevational zones (Appendix 2 Fig. \ref{fig:lam_zones}), though 41 species (52\%) exhibited elevational ranges that extended into both plateau and montane zones (Appendix 2 Table S4).

\begin{figure}
	\centering\includegraphics[width=5in]{../../../ms/1_Ecography/1/figs/DPCoA_Joint.png}
	\caption{\label{fig:dpcoa_joint} Double principle coordinate analysis (DPCoA) of local communities predicted by the \emph{Joint} model, with plots colored by (a) elevational bin, and (b) region. The central plateau includes hills from $\sim$300m to $\sim$1000m, with the Jura and the Alps rising steeply in the east and west, respectively. }
\end{figure}

Patterns of $\beta$-diversity among plots within each structured site did not strongly differ between models (Fig. \ref{fig:beta_div}). Overall $\beta$-diversity showed weak elevational patterns (\emph{Joint}: p=0.03, R$^2$=0.10, $\Delta$AICc=0.7; \emph{Structured}: p=0.06, R$^2$=0.06, $\Delta$AICc=3.1), representing relatively high variability among local communities on average regardless of elevation. The \emph{balanced variation} component drove variation among local plots at lower elevations, indicating turnover in the relative intensity of species such that species identities were likely to differ among plots (\emph{Joint}: p$<$0.001, R$^2$=0.70, $\Delta$AICc=5.4; \emph{Structured}: p$<$0.001, R$^2$=0.48, $\Delta$AICc=0.31). At higher elevations, the \emph{abundance gradient} component increased, indicating changes in total intensity among plots with less variation in species identities (\emph{Joint}: p$<$0.001, R$^2$=0.72, $\Delta$AICc=10.6; \emph{Structured}: p$<$0.001, R$^2$=0.44, $\Delta$AICc=2.4). 

\begin{figure}
\centering\includegraphics[width=3in]{../../../ms/1_Ecography/1/figs/beta_diversity.png}
\caption{\label{fig:beta_div} Multi-site $\beta$-diversity and its components across elevation. Posterior medians were used to calculate multi-site $\beta$-diversity and its components within each 1 km$^2$ site, representing the variation among plots. \emph{Balanced variation} quantifies changes in relative abundance among species, while \emph{abundance gradient} denotes changes in total abundance.}
\end{figure}


 




\section{Discussion}
\label{S:4}
Citizen science papers: \citep{Altwegg2019, Pernat2020, Henckel2020, Duan2020, Johnston2020,Robinson2020, Beck2010, Poisson2020}

% Summary
The \emph{Joint} model outperformed the \emph{Structured} model in predicting out-of-sample communities, indicating that the information provided by the presence-only data set at a regional scale improved predictions at a local scale as well. Further, the \emph{Joint} model reduced uncertainty in species-level responses at the regional scale, most dramatically for species that were not detected in the structured abundance data set. At a regional scale, ant richness was predicted to increase somewhat through the Swiss plateau, peak at the base of the mountain ranges, and then decline with further increases in elevation. In contrast, Shannon diversity was predicted to decline somewhat from low to high elevations. Patterns were best predicted by growing degree days, forest cover, and road length at a regional scale, and soil temperature, understory vegetation cover, and canopy cover at a local scale. Local ant communities cluster into plateau and montane communities, with distinct changes in the generic composition between zones. At low elevations, the variation among local communities is dominated by variation in relative abundance among species, while at high elevations it is more equally determined by variation in relative abundance and total abundance.

% Patterns
Divergent patterns at a local vs. regional scale. Seems like there's some density compensation \citep{LongColw2011}. The regional richness pattern fits with previous work, including compiled richness from species-level observed elevational ranges in the Alps \citep{Hellrigl2003,Glaser2006,SzewczykMcCain2016,Szewczyk2018}. Uncertainty is large at high elevations, partly due to extrapolation beyond the soil plots, and partly due to some of the species at higher elevations that occur at high densities and form super colonies. These species also showed a negative relationship with road length, and the high elevations include cells that require extrapolation beyond the covariate values available for fitting. Richness, diversity, and total intensity all tell somewhat different stories. Square kilometers are also pretty large, and particularly in the mountains can include a fairly large elevational range. They also don't take into account things like species interactions (though I'm skeptical that they actually scale up to a 1 km2) or stochastic variation in actual occupancy of available habitats. At the regional scale, the intensities are somewhat more about the potential occurrence I think. Using the average values doesn't account for the fact that each cell contains habitat that would be unsuitable for a given species, and therefore probably overestimates richness.

% Supported drivers
Paragraph on supported environmental drivers. Temperature preferences vary a lot between species, including among congeners (Appendix 2 Fig. \ref{fig:b_byParam}). So do effects of canopy, forest proportion, and road length. Species that show relationships with soil temperature or vegetation coverage tend to show positive relationships (Appendix 2 Fig. \ref{fig:b_bars}). That is, within a given (1 km$^2$) area, colonies are more likely to occur in local environments that are relatively warm and relatively productive, though the majority of species did not show a clear relationship. This fits with what we know about ants, since they tend to be thermophilic. The relationship of ant richness and diversity with habitat complexity, vegetation structure, or local productivity is unclear in the literature, with support and plausible hypotheses for both a positive and a negative relationship (CITE). Here, there are select genera or species that seem to show positive relationships, with no effect on the predicted intensity of most species. Specifically, many \emph{Myrmica} species, \emph{Solenopsis fugax}, and two \emph{Lasius} species (\emph{L. niger}, \emph{L. flavus}) are all more likely to occur in more productive microhabitats. Similarly, a relatively small number of species show strong positive or negative effects of canopy cover. Road length was mixed, and could represent diffuse human impacts or simply relatively open, relatively pebbly, relatively disturbed areas that are suitable or unsuitable for particular species.

% Latent variable and correlations
Paragraph speculating about what the latent variable and species correlation could represent. Probably more an unmeasured covariate than biotic interactions, looking at the correlation matrix. Maybe something that could affect the suitability for colony construction like the number of nice rocks or something. 

% Assemblages
Plateau ($<$ 1000 m) and montane ($>$ 1000 m) local communities are quite distinguishable based on species composition, and tend to form clear clusters (Fig. \ref{fig:dpcoa_joint}). There's some work that supported this, but it's not the most common thing to find based on my memory. However, this distinction does not mean that the communities are strongly Clementsian; 52\% of species were detected in both the plateau and the montane, with 33\% restricted to the plateau and 15\% exclusively montane. The distribution of species' predicted intensities similarly shows a varied distribution (Appendix 2 Fig. \ref{fig:lam_zones}). Thus, while the local communities can be clearly distinguished, that does not mean they are entirely distinct. Given the differences between the Jura and the Alps (which are what?), it's maybe a little surprising to not really see any differentiation. On the other hand, how determined are these lambdas by the environment? If the regional and local variables don't show much differentiation between the two mountain ranges, then the lambdas wouldn't either... I could add a PCA of the environmental variables to the appendix. But even so, there's nothing in the model that forces the communities to separate cleanly by elevation. That's a reflection of the data. I should also add an appendix figure of the elevational range of each species. There's also more local variation in the species composition at lower elevations (Fig. \ref{fig:beta_div}), meaning more variation in the expected species identities across plots. At higher elevations,the main difference among plots is instead the number of colonies expected to occur rather than which species are expected to occur.

% Hierarchical shrinkage and rare species
A key component of jointly estimated models such as the model here is that closely related species are assumed to respond somewhat similarly in the absence of compelling data. If there are no congeners (or no congeners with data), then a species is likely to respond in a manner similar to a generic ant, based on the aggregate responses. Thus, the \emph{Structured} model predicts similar responses among undetected genera. Nevertheless, the absolute intensity is still informed by the lack of observations in the structured samples. That is, undetected species are likely to be rare. This explains some of the differences in species-level and genus-level responses between the two models. For example, very few \emph{Camponotus} were detected in the structured samples (Appendix 2 Table S4) because they don't nest in the soil, and so the structured plots were unlikely to include a tree that served as a nest. Consequently, the richness for the genus predicted by the \emph{Structured} model is generally low, and the relative pattern is informed by the aggregate ($\beta$) responses as well as the limited detections of \emph{C. ligniperdus} (Appendix S2 Fig. \ref{fig:gen_map_Structured}, \ref{fig:b_byParam}). The pattern predicted by the \emph{Joint} model is rather different, because the preferences of these species are further informed by the occurrences from the presence-only data set (Appendix S2 Fig. \ref{fig:gen_map_Joint}, \ref{fig:b_byParam}). Thus, given a paucity of information on a species, the use of taxonomic hierarchical shrinkage provides a 'best guess' based on other species, with greater weight on closely related species, and with appropriately larger uncertainty (Fig. \ref{fig:slope_means}, Appendix S2 Fig. \ref{fig:b_byParam}).

% Bias in citizen science data
The sampling was definitely spatially biased, such that many cells only had one sample, and a few cells had hundreds. The taxonomic composition of the presence-only data set was not fully in line with the that of the structured samples, which more accurately quantified colony density, though the bias was as might be expected (Appendix S2 Fig. \ref{fig:D}). There was more of a tendency to under-represent species than over-represent species, particularly those that are not extremely rare such that they would be specifically sought after by contributing experts or enthusiasts, but they nevertheless have somewhat cryptic lifestyles or live in less-trafficked habitats like \emph{Myrmica}, some \emph{Lasius}, and some \emph{Temnothorax}. 

% Conclusion with a focus on the value of contribution of the citizen scientists
The efforts of citizen scientists in the canton of Vaud thus contributed valuable information not captured by the concurrent structured samples. The broad spatial coverage captured a correspondingly broad array of species, including rarer species that were unlikely to be detected in a structured survey focused on colony density. The \emph{Joint} model accounted for spatial and taxonomic bias, as well as geo-locational imprecision, to improve the estimation of species' responses to environmental variables as well as the predicted distribution of ants across the region. By leveraging the strengths of each collection effort, the hierarchical model described here provides a framework for utilizing the efforts of citizen scientists to improve our understanding of the distribution of biodiversity.




\section{Acknowledgments}
Thanks to everyone: main people with Opération Fourmis, expert taxonomists, citizen scientists, field help, feedback from lab group.

\newpage
\section{Bibliography}
\bibliography{../../../ms/opfo_diversity}



\newpage
\section{Appendix Descriptions}

\subsection{Appendix 1. Supplementary methods.}

\textbf{Table S1.} Covariate information

\textbf{Table S2.} Land cover types and canopy classification

\textbf{Table S3.} Prior distributions

\textbf{Code S1.} Stan code for full model



\subsection{Appendix 2. Supplementary results.}

\textbf{Table S4.} Species list with observed elevational range and counts

\textbf{Table S5.} Variable selection results

\textbf{Figure S1.} Species-level responses (points + HPDIs)

\textbf{Figure S2.} Species-level directional response summaries (bar proportions)

\textbf{Figure S3.} Taxonomic bias

\textbf{Figure S4.} Plateau vs montane species' intensities

\textbf{Figure S5.} DPCoA based on \emph{Structured} model predictions

\textbf{Figure S6.} Maps of richness (total, by SF, by genus)




\newpage

\section{Appendix 1}
\begin{table}[ht]
	\centering
	\begin{tabular}{ l l c }
		\hline
		\textbf{Parameter} & \textbf{Description} & \textbf{Type} \\
		\hline
		$i$ & structured sampling plots (1.2 m$^2$) & index \\
		$j$ & structured sampling cells (1 km$^2$) & index \\
		$k$ & citizen science cells (1 km$^2$) & index \\
		$s$ & species & index \\
		$g$ & genus & index \\
		$l$ & local covariates (1.2 m$^2$) & index \\
		$r$ & regional covariates (1 km$^2$) & index \\
		\hline
		$\mathbf{Y}_{is}$ & structured sampling counts (1.2 m$^2$) & data \\
		$\mathbf{W}_{ks}$ & citizen science counts (1 km$^2$) & data \\
		$\mathbf{V}_{il}$ & local covariates (1.2 m$^2$) & data \\
		$\mathbf{X}_{(jk)r}$ & regional covariates (1 km$^2$) & data \\
		$h$ & structured sampling proportional effort & data \\
		\hline
		$\mathbf{\lambda}_{is}$ & colony intensity (1.2 m$^2$) & latent \\
		$\mathbf{\Lambda}_{(jk)s}$ & colony intensity (1 km$^2$) & latent \\
		\hline
		$\alpha_{l}$ & aggregate ant responses (1.2 m$^2$) & slopes \\
		$\mathbf{A}_{lg}$ & genus-level ant responses (1.2 m$^2$) & slopes \\
		$\mathbf{a}_{ls}$ & species-level ant responses (1.2 m$^2$) & slopes \\
		$\sigma^a_{l}$ & response sd among congeners & sd \\
		$\mathbf{\Sigma^A}_{gg}$ & genus-level covariance matrix & cov mx \\
		$\beta_{r}$ & aggregate ant responses (1 km$^2$) & slopes \\
		$\mathbf{B}_{rg}$ & genus-level ant responses (1 km$^2$) & slopes \\
		$\mathbf{b}_{rs}$ & species-level ant responses (1 km$^2$) & slopes \\
		$\sigma^b_{r}$ & response sd among congeners & sd \\
		$\mathbf{\Sigma^B}_{gg}$ & genus-level covariance matrix & cov mx \\
		$D_{s}$ & citizen science species bias (proportional) & random effect \\
	\end{tabular}
	\caption{\label{table:params} Parameters in the model. Could be moved to an appendix, or shortened since many of these don't need to be highlighted. }
\end{table}






\newpage

\section{Appendix 2}

\begin{figure}
	\centering\includegraphics[height=7.5in]{../../../ms/1_Ecography/1/figs/b_opt_byParam.png}
	\caption{\label{fig:b_byParam} Posterior species responses to local and regional covariates. Points show posterior medians, while lines show the Highest Posterior Density Intervals (HPDIs: 80\%: thick; 95\%: thin). Solid points indicate 95\% HPDIs that exclude zero in the \emph{Joint} (purple) and \emph{Structured} (green) models.}
\end{figure}

\begin{figure}
	\centering\includegraphics[width=6.5in]{../../../ms/1_Ecography/1/figs/b_opt_bar.png}
	\caption{\label{fig:b_bars} Proportion of species' responses to each covariate. Responses are considered positive (red) or negative (blue) if the Highest Posterior Density Interval (HPDI) excludes zero for (a) 95\% HPDIs and (b) 80\% HPDIs. }
\end{figure}

\begin{figure}
	\centering\includegraphics[height=8in]{../../../ms/1_Ecography/1/figs/lambda_zones.png}
	\caption{\label{fig:lam_zones} Posterior species intensity across plateau and montane zones in each model. Each species' total intensity (a. Regional; b. Local) was partitioned into the proportion occurring in plateau ($<$1000m: grey) and montane ($\geq$1000m: green) elevations. Vertical lines indicate 80\%, 90\%, and 95\% for either zone. }
\end{figure}

\begin{figure}
	\centering\includegraphics[width=5in]{../../../ms/1_Ecography/1/figs/DPCoA_Struc.png}
	\caption{\label{fig:dpcoa_struc} Double principle coordinate analysis (DPCoA) of local communities predicted by the \emph{Structured} model, with plots colored by (a) elevational bin, and (b) region. The central plateau includes hills from $\sim$300m to $\sim$1000m, with the Jura and the Alps rising steeply in the east and west, respectively. }
\end{figure}

\begin{figure}
	\centering\includegraphics[height=8in]{../../../ms/1_Ecography/1/figs/D.png}
	\caption{\label{fig:D} Proportional taxonomic bias posterior distributions. Medians and Highest Posterior Density Intervals (HPDIs; 80\%: thick; 95\%: thin) for the representation of each species in the presence-only data set relative to the structured abundance data set. Values less than 1 indicate under-representation in the presence-only data set. Dark red: 95\% HPDIs exclude zero; light red: 80\% HPDIs exclude zero. }
\end{figure}

\begin{figure}
	\centering\includegraphics[width=5in]{../../../ms/1_Ecography/1/figs/maps/sf_S_J.png}
	\caption{\label{fig:sf_map_Joint} Maps of predicted richness in the \emph{Joint} model for subfamilies. Predictions are based on 95\% Highest Posterior Density Intervals for each species. }
\end{figure}


\begin{figure}
	\centering\includegraphics[width=5in]{../../../ms/1_Ecography/1/figs/maps/sf_S_S.png}
	\caption{\label{fig:sf_map_Structured+LV} Maps of predicted richness in the \emph{Structured} model for subfamilies. Predictions are based on 95\% Highest Posterior Density Intervals for each species. }
\end{figure}

\begin{figure}
	\centering\includegraphics[width=6in]{../../../ms/1_Ecography/1/figs/maps/gen_S_J.png}
	\caption{\label{fig:gen_map_Joint} Maps of predicted richness in the \emph{Joint} model for genera with at least two species. Predictions are based on 95\% Highest Posterior Density Intervals for each species. }
\end{figure}

\begin{figure}
	\centering\includegraphics[width=6in]{../../../ms/1_Ecography/1/figs/maps/gen_S_S.png}
	\caption{\label{fig:gen_map_Structured} Maps of predicted richness in the \emph{Structured} model for genera with at least two species. Predictions are based on 95\% Highest Posterior Density Intervals for each species. }
\end{figure}







\end{document}
